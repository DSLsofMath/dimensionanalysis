% -*-Latex-*-
\RequirePackage{amsmath}
\documentclass{article}
\usepackage[utf8]{inputenc}
\usepackage[T1]{fontenc}
\usepackage{graphicx}
\usepackage{hyperref}
\usepackage{newunicodechar}
\newcommand{\tikzstar}[5]{% inner radius, outer radius, tips, rot angle, options
\begin{tikzpicture}[baseline={([yshift=-\axisht]current bounding box.center)},x=1ex,y=1ex,line width=.1ex]
  \pgfmathsetmacro{\starangle}{360/#3}
  \draw[#5] (#4:#1)
  \foreach \x in {1,...,#3}
  { -- (#4+\x*\starangle-\starangle/2:#2) -- (#4+\x*\starangle:#1)
  }
  -- cycle;
\end{tikzpicture}
}

\newcommand{\sqdiamond}{\mathbin{\text{\tikz [x=1ex,y=1ex,line width=.1ex,line join=round] \draw (0,0) rectangle (1,1) (0,.5) -- (.5,1) -- (1,.5) -- (.5,0) -- (0,.5) -- cycle;}}}
\newcommand{\varsqdiamond}{\mathbin{\text{\tikz [x=1ex,y=1ex,line width=.1ex,line join=round] \draw (0,0) rectangle (1,1) (.5\pgflinewidth,.5) -- (.5,1ex-.5\pgflinewidth) -- (1ex-.5\pgflinewidth,.5) -- (.5,.5\pgflinewidth) -- (.5\pgflinewidth,.5) -- cycle;}}}
\newcommand{\varsquare}{\mathbin{\text{\tikz [x=1ex,y=1ex,line width=.1ex,line join=round] \draw (0,0) rectangle (1,1);}}}
\newcommand{\smalltriangleleft}{\mathbin{\text{\tikz [baseline={([yshift=-\axisht]current bounding box.center)},x=1ex,y=1ex,line width=.1ex] \draw (0,0.3) -- (-0.52,0) -- (0,-0.3) -- cycle;}}}
\def\axisht{\dimexpr.5\fontcharht\font`)-.5\fontchardp\font`)\relax} % https://tex.stackexchange.com/questions/386288

% \newunicodechar{ϕ}{\ensuremath{\mathnormal{\varphi}}} % 
\newunicodechar{ }{\,}
\newunicodechar{¬}{\ensuremath{\neg}} % 
\newunicodechar{²}{^2}
\newunicodechar{³}{^3}
\newunicodechar{·}{\ensuremath{\cdot}}
\newunicodechar{¹}{^1}
\newunicodechar{×}{\ensuremath{\times}} % 
\newunicodechar{÷}{\ensuremath{\div}} % 
\newunicodechar{ʲ}{^j}
\newunicodechar{ʳ}{^r}
\newunicodechar{ˡ}{^l}
\newunicodechar{ˢ}{^s}
\newunicodechar{̷}{\not} % 
\newunicodechar{Γ}{\ensuremath{\Gamma}}   %
\newunicodechar{Δ}{\ensuremath{\Delta}} % 
\newunicodechar{Η}{\ensuremath{\textrm{H}}} % 
\newunicodechar{Θ}{\ensuremath{\Theta}} % 
\newunicodechar{Λ}{\ensuremath{\Lambda}} % 
\newunicodechar{Ξ}{\ensuremath{\Xi}} % 
\newunicodechar{Π}{\ensuremath{\Pi}}   %
\newunicodechar{Σ}{\ensuremath{\Sigma}} % 
\newunicodechar{Φ}{\ensuremath{\Phi}} % 
\newunicodechar{Ψ}{\ensuremath{\Psi}} % 
\newunicodechar{Ω}{\ensuremath{\Omega}} % 
\newunicodechar{α}{\ensuremath{\mathnormal{\alpha}}}
\newunicodechar{β}{\ensuremath{\beta}} % 
\newunicodechar{γ}{\ensuremath{\mathnormal{\gamma}}} % 
\newunicodechar{δ}{\ensuremath{\mathnormal{\delta}}} % 
\newunicodechar{ε}{\ensuremath{\mathnormal{\varepsilon}}} % 
\newunicodechar{ζ}{\ensuremath{\mathnormal{\zeta}}} % 
\newunicodechar{η}{\ensuremath{\mathnormal{\eta}}} % 
\newunicodechar{θ}{\ensuremath{\mathnormal{\theta}}} % 
\newunicodechar{ι}{\ensuremath{\mathnormal{\iota}}} % 
\newunicodechar{κ}{\ensuremath{\mathnormal{\kappa}}} % 
\newunicodechar{λ}{\ensuremath{\mathnormal{\lambda}}} % 
\newunicodechar{μ}{\ensuremath{\mathnormal{\mu}}} % 
\newunicodechar{ν}{\ensuremath{\mathnormal{\nu}}} % 
\newunicodechar{ξ}{\ensuremath{\mathnormal{\xi}}} % 
\newunicodechar{π}{\ensuremath{\mathnormal{\pi}}}
\newunicodechar{ρ}{\ensuremath{\mathnormal{\rho}}} % 
\newunicodechar{σ}{\ensuremath{\mathnormal{\sigma}}} % 
\newunicodechar{τ}{\ensuremath{\mathnormal{\tau}}} % 
\newunicodechar{φ}{\ensuremath{\mathnormal{\phi}}} % 
\newunicodechar{χ}{\ensuremath{\mathnormal{\chi}}} % 
\newunicodechar{ψ}{\ensuremath{\mathnormal{\psi}}} % 
\newunicodechar{ω}{\ensuremath{\mathnormal{\omega}}} %  
\newunicodechar{ϕ}{\ensuremath{\mathnormal{\phi}}} % 
\newunicodechar{ϵ}{\ensuremath{\mathnormal{\epsilon}}} % 
\newunicodechar{ᵀ}{\ensuremath{^T}}
\newunicodechar{ᵏ}{^k}
\newunicodechar{ᵐ}{^m}
\newunicodechar{ᵢ}{\ensuremath{_i}} % 
\newunicodechar{ᵣ}{_r}
\newunicodechar{ᶿ}{^\theta}
\newunicodechar{ }{\quad}
\newunicodechar{†}{\dagger}
\newunicodechar{ }{\,}
\newunicodechar{′}{\ensuremath{^\prime}}  % 
\newunicodechar{″}{\ensuremath{^\second}} % 
\newunicodechar{‴}{\ensuremath{^\third}}  % 
\newunicodechar{ⁱ}{^i}
\newunicodechar{⁵}{\ensuremath{^5}}
\newunicodechar{⁺}{\ensuremath{^+}} %% \newunicodechar{⁺}{^+}
\newunicodechar{⁻}{\ensuremath{^-}} %% 
\newunicodechar{ⁿ}{^n}
\newunicodechar{₀}{\ensuremath{_0}} % 
\newunicodechar{₁}{\ensuremath{_1}} % \newunicodechar{₁}{_1}
\newunicodechar{₂}{\ensuremath{_2}} % \newunicodechar{₂}{_2}
\newunicodechar{₃}{\ensuremath{_3}}
\newunicodechar{₊}{\ensuremath{_+}} %% 
\newunicodechar{₋}{\ensuremath{_-}} %% 
\newunicodechar{ₖ}{\ensuremath{_k}} % 
\newunicodechar{ₗ}{\ensuremath{_l}}
\newunicodechar{ₘ}{_m}
\newunicodechar{ₙ}{_n} % 
\newunicodechar{ₛ}{_s}
\newunicodechar{⃰}{\ensuremath{^{\ast}}}
\newunicodechar{ℂ}{\ensuremath{\mathbb{C}}} % 
\newunicodechar{ℒ}{\ensuremath{\mathscr{L}}}
\newunicodechar{ℕ}{\mathbb{N}} % 
\newunicodechar{ℚ}{\ensuremath{\mathbb{Q}}}
\newunicodechar{ℝ}{\ensuremath{\mathbb{R}}} % 
\newunicodechar{ℤ}{\ensuremath{\mathbb{Z}}} % 
\newunicodechar{ℳ}{\mathscr{M}}
\newunicodechar{⅋}{\ensuremath{\parr}} % 
\newunicodechar{←}{\ensuremath{\leftarrow}} % 
\newunicodechar{↑}{\ensuremath{\uparrow}} % 
\newunicodechar{→}{\ensuremath{\rightarrow}} %
\newunicodechar{↔}{\ensuremath{\leftrightarrow}} % 
\newunicodechar{↖}{\nwarrow} %
\newunicodechar{↗}{\nearrow} %
\newunicodechar{↝}{\ensuremath{\leadsto}}
\newunicodechar{↦}{\ensuremath{\mapsto}}
\newunicodechar{⇆}{\ensuremath{\leftrightarrows}} % 
\newunicodechar{⇐}{\ensuremath{\Leftarrow}} % 
\newunicodechar{⇒}{\ensuremath{\Rightarrow}} % 
\newunicodechar{⇔}{\ensuremath{\Leftrightarrow}} % 
\newunicodechar{∀}{\ensuremath{\forall}}   %
\newunicodechar{∂}{\ensuremath{\partial}}
\newunicodechar{∃}{\ensuremath{\exists}} % 
\newunicodechar{∅}{\ensuremath{\varnothing}} % 
\newunicodechar{∇}{\ensuremath{\nabla}} % 
\newunicodechar{∈}{\ensuremath{\in}}
\newunicodechar{∉}{\ensuremath{\not\in}} % 
\newunicodechar{∋}{\ensuremath{\ni}}  % 
\newunicodechar{∎}{\ensuremath{\qed}}%\newunicodechar{∎}{\ensuremath{\blacksquare}} % end of proof
\newunicodechar{∏}{\prod}
\newunicodechar{∑}{\sum}
\newunicodechar{∗}{\ensuremath{\ast}} % 
\newunicodechar{∘}{\ensuremath{\circ}} % 
\newunicodechar{∙}{\ensuremath{\bullet}} % \newunicodechar{∙}{\ensuremath{\cdot}}
\newunicodechar{∝}{\propto}
\newunicodechar{∞}{\ensuremath{\infty}} % 
\newunicodechar{∣}{\ensuremath{\mid}} % 
\newunicodechar{∧}{\ensuremath{\wedge}}% 
\newunicodechar{∨}{\vee}% 
\newunicodechar{∩}{\ensuremath{\cap}} % 
\newunicodechar{∪}{\ensuremath{\cup}} % 
\newunicodechar{∫}{\int}
\newunicodechar{∷}{::} % 
\newunicodechar{∼}{\ensuremath{\sim}} % 
\newunicodechar{≃}{\ensuremath{\simeq}} % 
\newunicodechar{≅}{\ensuremath{\cong}} % 
\newunicodechar{≈}{\ensuremath{\approx}} % 
\newunicodechar{≜}{\ensuremath{\stackrel{\scriptscriptstyle {\triangle}}{=}}} % \newunicodechar{≜}{\triangleq}
\newunicodechar{≝}{\ensuremath{\stackrel{\scriptscriptstyle {\text{def}}}{=}}}
\newunicodechar{≟}{\ensuremath{\stackrel {_\text{\textbf{?}}}{\text{\textbf{=}}\negthickspace\negthickspace\text{\textbf{=}}}}} % or {\ensuremath{\stackrel{\scriptscriptstyle ?}{=}}}
\newunicodechar{≠}{\ensuremath{\neq}}% 
\newunicodechar{≡}{\ensuremath{\equiv}}% 
\newunicodechar{≤}{\ensuremath{\le}} % 
\newunicodechar{≥}{\ensuremath{\ge}} % 
\newunicodechar{⊂}{\ensuremath{\subset}} % 
\newunicodechar{⊃}{\ensuremath{\supset}} %  
\newunicodechar{⊆}{\ensuremath{\subseteq}} %  
\newunicodechar{⊇}{\ensuremath{\supseteq}} % 
\newunicodechar{⊎}{\ensuremath{\uplus}} %
\newunicodechar{⊑}{\ensuremath{\sqsubseteq}} % 
\newunicodechar{⊒}{\ensuremath{\sqsupseteq}} % 
\newunicodechar{⊓}{\ensuremath{\sqcap}} % 
\newunicodechar{⊔}{\ensuremath{\sqcup}} % 
\newunicodechar{⊕}{\ensuremath{\oplus}} % 
\newunicodechar{⊗}{\ensuremath{\otimes}} % 
\newunicodechar{⊙}{\ensuremath{\odot}} % 
\newunicodechar{⊛}{\ensuremath{\circledast}}
\newunicodechar{⊢}{\ensuremath{\vdash}} % 
\newunicodechar{⊤}{\ensuremath{\top}}
\newunicodechar{⊥}{\ensuremath{\bot}} % bottom
\newunicodechar{⊧}{\models} % 
\newunicodechar{⊨}{\models} % 
\newunicodechar{⊩}{\Vdash}
\newunicodechar{⊬}{\not \vdash}
\newunicodechar{⊸}{\ensuremath{\multimap}} % 
\newunicodechar{⋁}{\ensuremath{\bigvee}}
\newunicodechar{⋃}{\ensuremath{\bigcup}} % 
\newunicodechar{⋄}{\ensuremath{\diamond}} % 
\newunicodechar{⋅}{\ensuremath{\cdot}}
\newunicodechar{⋆}{\ensuremath{\star}} % 
\newunicodechar{⋮}{\ensuremath{\vdots}} % 
\newunicodechar{⋯}{\ensuremath{\cdots}} % 
\newunicodechar{─}{---}
\newunicodechar{■}{\ensuremath{\blacksquare}} % black square
\newunicodechar{□}{\ensuremath{\varsquare}} % 
\newunicodechar{▴}{\ensuremath{\blacktriangledown}}
\newunicodechar{▵}{\ensuremath{\triangle}}
\newunicodechar{▹}{\ensuremath{\triangleright}} % 
\newunicodechar{▻}{\ensuremath{\rhd}} % 
\newunicodechar{▾}{\ensuremath{\blacktriangle}}
\newunicodechar{▿}{\ensuremath{\triangledown}}
\newunicodechar{◃}{\ensuremath{\triangleleft}}
\newunicodechar{◅}{\ensuremath{\lhd}}
\newunicodechar{◇}{\ensuremath{\diamond}} % 
\newunicodechar{◽}{\ensuremath{\varsquare}}
\newunicodechar{★}{\ensuremath{\star}}   %
\newunicodechar{♭}{\ensuremath{\flat}} % 
\newunicodechar{♯}{\ensuremath{\sharp}} % 
\newunicodechar{⛋}{\ensuremath{\varsqdiamond}} % 
\newunicodechar{✓}{\ensuremath{\checkmark}} % 
\newunicodechar{⟂}{\ensuremath{^\bot}} % PERPENDICULAR 
\newunicodechar{⟦}{\ensuremath{\llbracket}}
\newunicodechar{⟧}{\ensuremath{\rrbracket}}
\newunicodechar{⟨}{\ensuremath{\langle}} % 
\newunicodechar{⟩}{\ensuremath{\rangle}} % 
\newunicodechar{⟶}{{\longrightarrow}}
\newunicodechar{⟷} {\ensuremath{\leftrightarrow}}
\newunicodechar{⟹}{\ensuremath{\Longrightarrow}} % 
\newunicodechar{⦇}{\rlparenthesis}
\newunicodechar{⦈}{\llparenthesis}
\newunicodechar{⪅}{\ensuremath{\lessapprox}}
\newunicodechar{⪆}{\ensuremath{\gtrapprox}}
\newunicodechar{ⱼ}{\ensuremath{_j}} % 
\newunicodechar{𝒟}{\ensuremath{\mathcal{D}}} % 
\newunicodechar{𝒢}{\ensuremath{\mathcal{G}}}
\newunicodechar{𝒦}{\ensuremath{\mathcal{K}}} % 
\newunicodechar{𝒫}{\ensuremath{\mathcal{P}}}
\newunicodechar{𝔸}{\ensuremath{\mathds{A}}} % 
\newunicodechar{𝔹}{\ensuremath{\mathds{B}}} % 
\newunicodechar{𝔼}{\mathds{E}}
\newunicodechar{𝟙}{\ensuremath{\mathds{1}}} % 

\author{Patrik Jansson}
\date{\today}
\title{Dimensional Analysis and Graded Algebras}
\hypersetup{
 pdfauthor={Patrik Jansson},
 pdftitle={Dimensional Analysis and Graded Algebras},
 pdfkeywords={},
 pdfsubject={},
 pdflang={English}}
%include lhs2TeX.fmt
%include lhs2TeX.sty
%include polycode.fmt
% %format / = "\mathbin{/}"
%format : = "\mathop{:}"
%format -> = "\,\to\,"
\newcommand{\un}{\mbox{\texttt{\_}}}   % short underscore
\newcommand{\unopun}[1]{\ensuremath{\,\un{#1}\un\,}}
%format _≡_ = "\unopun{" ≡ "}"
%format _==_ = "\unopun{" == "}"
%format ^ = "\mathbin{\!\hat{}\!}"
%format _*d_ = "\unopun{" *d "}"
%format *d = "\mathbin{{*}_d}"
%format _/d_ = "\unopun{" /d "}"
%format /d = "\mathbin{{/\!}_d}"
%format _^d_ = "\unopun{" ^d "}"
%format ^d = "\mathbin{\hat{\;}^d}"
%format _+q_ = "\unopun{" +q "}"
%format +q = "\mathbin{{+\!}_q}"
%format _*q_ = "\unopun{" *q "}"
%format *q = "\mathbin{{*}_q}"
%format _/q_ = "\unopun{" /q "}"
%format /q = "\mathbin{{/\!}_q}"
%format _^q_ = "\unopun{" ^q "}"
%format ^q = "\mathbin{\hat{\;}^q}"
%format scaleq = scale "_q"
%format _+v_ = "\unopun{" +v "}"
%format +v = "\mathbin{{+\!}_v}"
%format _-v_ = "\unopun{" -v "}"
%format -v = "\mathbin{{-\!}_v}"
%format _+s_ = "\unopun{" +s "}"
%format +s = "\mathbin{{+\!}_s}"
%format _*s_ = "\unopun{" *s "}"
%format *s = "\mathbin{{*\!}_s}"
%format _/s_ = "\unopun{" /s "}"
%format /s = "\mathbin{{/\!}_s}"
%format d1 = "d_1"
%format d2 = "d_2"
%format q0 = 0 "_{" q "}"
%format q1 = 1 "_{" q "}"
%format s0 = 0 "_{" s "}"
%format s1 = 1 "_{" s "}"
%format ℤ0 = 0 "_{" ℤ "}"
%format ℤ1 = 1 "_{" ℤ "}"
%format ℤ2 = 2 "_{" ℤ "}"
% the `doubleequals' macro is due to Jeremy Gibbons
\def\doubleequals{\mathrel{\unitlength 0.01em
  \begin{picture}(78,40)
    \put(7,34){\line(1,0){25}} \put(45,34){\line(1,0){25}}
    \put(7,14){\line(1,0){25}} \put(45,14){\line(1,0){25}}
  \end{picture}}}
%% If you remove the %format == command the lhs2TeX default yields ≡, which can be a problem
%format ==   = "\doubleequals"
\begin{document}
\section{Patrik Jansson: Dimension analysis and graded algebras}
\label{sec:org7abdf0f}
\begin{itemize}
\item Prel. talk 2023-02-21
\item at the \href{https://ifipwg21wiki.cs.kuleuven.be/IFIP21/OnlineFeb23}{online meeting} of WG 2.1 on Algorithmic Languages and Calculi
\item Joint work with Nicola Botta and Guilherme Silva.
\item Abstract:
This talk describes work in progress aimed at understanding dimension analysis
though coding it up using dependent types (in Agda). I will talk you through
definitions of physical quantities, systems of units, dimensions, dimensional
analysis, and an example of applying it to modelling a simple pendulum
experiment.
\item Some history: I did an Agda tutorial at the WG2.1 meeting \#63 in Kyoto
(in 2007) and I found mentions of Agda in 8 other physical meetings after
that.
\item Source code: \url{https://github.com/DSLsofMath/dimensionanalysis}
\end{itemize}
%if False
\subsection{Agda version etc.}
\label{sec:org0a55a2b}
\begin{itemize}
\item This file loads in Agda 2.6.2.2 with Agda stdlib-1.7
\item It is a literate Agda file with embedded LaTeX code.
\end{itemize}
%endif
%if False
\subsection{Skip some imports and setup}
\label{sec:orga01e249}
\subsubsection{Basic imports: equality, integers}
\label{sec:org28c6c1b}
\begin{code}
module JanssonDimensions2023 where
open Agda.Primitive using (lzero)
open import Relation.Binary.PropositionalEquality renaming (_≡_ to _==_)
open import Relation.Nullary renaming (¬_ to Not)
Type = Set
Type1 = Set1

import Data.Integer using (ℤ; +0; +_; -[1+_])
                    renaming (0ℤ to ℤ0; 1ℤ to ℤ1)
open Data.Integer
Integer : Type
Integer = ℤ
ℤ2 : ℤ
ℤ2 = + 2
\end{code}
\subsubsection{Lift |Ring| operations to vectors}
\label{sec:org1dcf010}
\begin{code}
import Algebra
Ring = Algebra.Ring lzero lzero
open import Data.Nat hiding (_^_) renaming (ℕ to Nat; _+_ to _+n_; _*_ to _*n_)
open import Data.Vec renaming (foldr to depFoldr)
foldr : ∀ {A : Type} {B : Type} {m} →
        (A → B → B) → B → Vec A m → B
foldr {A} {B} op = depFoldr (\ _ -> B) (\{m} -> op)
module VectScope (r : Ring) where

  A = Algebra.Ring.Carrier r
  open Algebra.Ring r

  0v : {n : Nat} -> Vec A n
  0v = replicate 0#

  _+v_  :  {n : Nat} ->
           Vec A n -> Vec A n -> Vec A n
  _+v_  =  zipWith _+_

  _-v_  :  {n : Nat} ->
           Vec A n -> Vec A n -> Vec A n
  _-v_  =  zipWith _-_

  _*v_  :  {n : Nat} ->
           A -> Vec A n -> Vec A n
  x *v v =  map (x *_) v
\end{code}
\subsubsection{Postulate a Ring of reals}
\label{sec:org65e7a7d}
\begin{itemize}
\item implemented using |module Data.Float| and some postulates
\item \textbf{NOTE} - this is not true but used for pragmatic reasons
\end{itemize}
\begin{code}
import Data.Float using (Float)
Real : Type
Real = Data.Float.Float

import Data.Float.Base

module RealRingPostulates where
  Carrier = Real
  _≈_ = _==_
  _+_ = Data.Float.Base._+_
  _*_ = Data.Float.Base._*_
  -_  = Data.Float.Base.-_
  0#  = 0.0
  1#  = 1.0
  postulate isRing : Algebra.IsRing _≈_ _+_ _*_ -_ 0# 1#

RealRing : Ring
RealRing = record { RealRingPostulates }
open Data.Float.Base renaming (_*_ to _*r_; _+_ to _+r_; _-_ to _-r_; _÷_ to _/r_)
\end{code}
\subsubsection{Raw Field record type + Real field record}
\label{sec:org04f9479}
\begin{code}
record RawField (S : Type) : Type where
  -- TODO redo with reuse of Ring structure?
  field
    s0 s1 : S
    _+s_ : S -> S -> S
    _*s_ : S -> S -> S
    _/s_ : S -> S -> S

  _^_ : S → Nat → S
  s ^ zero   = s1
  s ^ suc n  = (s ^ n) *s s

  pow : S -> ℤ -> S
  pow s  ( + n )  =        s ^ n
  pow s -[1+ n ]  = s1 /s (s ^ (suc n))

rfReal : RawField Real
rfReal = record {s0 = 0.0; s1 = 1.0; _+s_ = _+r_; _*s_ = _*r_; _/s_ = _/r_}

RealPlus3 : Type
RealPlus3 = Vec Real 3
\end{code}
%endif
\subsection{Physical quantities and systems of units}
\label{sec:orgb97e3b3}
\begin{itemize}
\item We can assign a "dimension" to each physical quantity:
informally |dim : Q -> D| (later an indexed type |Q d|).
\item Physical quantities like length, speed, force, etc. are usually
measured with respect to a system of units of measurement, one unit
per "base dimension".
\end{itemize}
\subsubsection{These systems can be grouped into classes}
\label{sec:org2870a1b}
\begin{itemize}
\item For geometry, just one base dimension of length is needed and the
class is usually called just L.
\item For kinematics (the class LT) we have a length and a time.
\item For mechancics (the class LTM) we have length, time, and mass.
\item Etc.
\end{itemize}
\subsection{Dimensions and dimension functions}
\label{sec:org5f28f0f}
\begin{itemize}
\item Usually dimensions are decribed by monomials: acceleration "has
dimension |L / T ^ 2|". The formal reading is that the dimension
\textbf{function} is |\L T M -> L / T ^ 2|. This function describes how the
measured value of the acceleration changes when we change the units
of measurements.
\item If we start with a simpler case it is easier to see: if we measure
my height (of dimension L) in meters the value is 1.78 but if we
divide the unit by 100 (to get cm) my height is measured to 178.
\item In general, if we make the unit of measurement L times smaller, we
make the measured height L times bigger.
\item This is usually described as saying that the height is actually
invariant, but the measured value changes in the opposite direction
of the measuring rod.
\item This simplest (linear scaling) relation holds for quantities of one
of the base dimensions, but changes if we move to derived dimensions
like that for area or acceleration.
\item In general, if we make the unit of length |L| times smaller, the
unit of time |T| times smaller and the unit of mass |M| times
smaller, the measuread acceleration becomes |L / T ^ 2| times bigger and
the measured force |M * L / T ^ 2| times bigger.
\end{itemize}
\subsection{Physics and dimensions}
\label{sec:orgfce6857}
\begin{itemize}
\item In an equation like |F = m * a| (force equals mass times
acceleration) in physics, the dimensions must match up:
\begin{spec}
dim F = dim (m * a)
\end{spec}
\item Similarly for addition.
\item For multiplication we don't need to require matching
dimension: |dim| is a homomorphism: |dim (m * a) = dim m *d dim a|.
\item Now, how do we type this? We clearly need to be able to multiply and
divide dimensions, and we also need a value to denote
"dimensionless".
\end{itemize}
\subsubsection{DimStuff}
\label{sec:orgb0d530a}
\begin{code}
record DimStuff : Type1 where
  field
    Dim : Type
    Dimless : Dim
    _*d_  : Dim -> Dim -> Dim
    _/d_  : Dim -> Dim -> Dim

  _^d_ : Dim -> ℤ -> Dim
\end{code}
(We will get back to laws later, and to concrete implementations.)
%if False
\subsubsection{Skip details (exponentiation)}
\label{sec:org8b0821f}
\begin{code}
  _^dn_ : Dim -> Nat -> Dim
  d ^dn zero   = Dimless
  d ^dn suc n  = (d ^dn n) *d d

  d ^d  ( + n )  = d ^dn n
  d ^d -[1+ n ]  = Dimless /d (d ^dn suc n)
\end{code}
%endif
\subsection{Quantities indexed by dimensions}
\label{sec:org0cd7662}
Then we need to make sure the type for "quantities" is indexed by
dimensions. We assume a type for dimensions, and a type of
scalars |S|.
\begin{code}
module Quantities (dimstuff : DimStuff) (S : Type) where
  open DimStuff dimstuff

  variable d d1 d2 : Dim
  record Qstuff : Type1 where
    field
      Q     : Dim -> Type
      dim   : Q d -> Dim

      q0      : Q d
      _+q_    : Q d  -> Q d  -> Q d
      scaleq  : S    -> Q d  -> Q d
\end{code}
\subsection{Quantities are almost a field}
\label{sec:org0c728a2}
\begin{code}
      q1    : Q Dimless
      _*q_  : Q d1 -> Q d2 -> Q (d1 *d d2)
      _/q_  : Q d1 -> Q d2 -> Q (d1 /d d2)
\end{code}

Here we see that the value of the dimension index tracks the
operations performed on the quantities.

%if False
\subsubsection{Skip (fixities, exponentiation)}
\label{sec:orgd136fdd}
\begin{code}
    infixl 7 _*q_
    infixl 7 _/q_
    infixl 6 _+q_

    _^qn_ : Q d -> (n : Nat) -> Q (d ^dn n)
    s ^qn zero   = q1
    s ^qn suc n  = (s ^qn n) *q s

    _^q_ : Q d -> (i : ℤ) -> Q (d ^d i)
    s ^q  ( + n )  = s ^qn n
    s ^q -[1+ n ]  = q1 /q (s ^qn (suc n))
  open Qstuff
\end{code}
%endif
\subsection{Dimension structure}
\label{sec:orgd044122}
The algebraic structures at play for the dimensions part is a group:
and we can use a vector of integers to keep track of the exponents of
the base dimensions:

\subsection{Concrete example: the LTM class}
\label{sec:org58f9714}
%if False
\subsubsection{Skip module header / imports}
\label{sec:orgc1be1b5}
\begin{code}
module LTM (S : Type) (rf : RawField S) where
  open RawField rf public
  module LTMDim where
    open import Data.Integer.Properties using (+-*-ring) public
    open module V = VectScope +-*-ring public
\end{code}
%endif
\subsubsection{LTM Dim implementation example}
\label{sec:orgef10fd0}
\begin{code}
    Dim = Vec ℤ 3
    _*d_ = _+v_
    _/d_ = _-v_
    Dimless : Dim
    Dimless  = ℤ0 ∷ ℤ0 ∷ ℤ0 ∷ [] -- zero vector
    L T M : Dim -- base vectors / dimensions
    L        = ℤ1 ∷ ℤ0 ∷ ℤ0 ∷ []
    T        = ℤ0 ∷ ℤ1 ∷ ℤ0 ∷ []
    M        = ℤ0 ∷ ℤ0 ∷ ℤ1 ∷ []
\end{code}
\subsection{Quantities}
\label{sec:org52ece7d}
%if False
\subsubsection{Skip details (module end + opens)}
\label{sec:org22e4838}
\begin{code}
  dimstuff : DimStuff
  dimstuff = record { LTMDim }

  open Quantities dimstuff S public
  open LTMDim
\end{code}
%endif
\subsubsection{|Q| record type: just a wrapper around a scalar value}
\label{sec:org235f1bd}
\begin{code}
  module LTMQ where
    record Q (d : Dim) : Type where
      constructor Val
      field
        val : S
    open Q public
    dim : Q d -> Dim
    dim {d} _ = d
\end{code}
\subsubsection{|Q d| is a 1D vector space for each |d : Dim|}
\label{sec:orgf183c4d}
\begin{code}
    q0 : Q d
    q0 = Val s0
    _+q_ : Q d  -> Q d  -> Q d
    Val x +q Val y = Val (x +s y)
    scaleq : S -> Q d -> Q d
    scaleq s (Val x) = Val (s *s x)
\end{code}
\subsubsection{|Q| is a graded field (informally - no proofs)}
\label{sec:orgb5405e9}
\begin{code}
    q1  : Q Dimless
    q1 = Val s1
    _*q_ : Q d1 -> Q d2 -> Q (d1 *d d2)
    Val x *q Val y = Val (x *s y)
    _/q_ : Q d1 -> Q d2 -> Q (d1 /d d2)
    Val x /q Val y = Val (x /s y)
\end{code}
%if False
\subsubsection{Skip (module end, open)}
\label{sec:orgcc8d37a}
\begin{code}
  qstuff : Qstuff
  qstuff = record { LTMQ }
  open Qstuff qstuff
\end{code}
%endif

\subsection{Discussion: field or not a field?}
\label{sec:orgb7e65ef}
For the scalars we need an underlying field |S| and we get a field
back for |Q Dimless| but not for |Q d| for other |d|. But we get (yet
another) example of an indexed structure: an indexed, or graded, field.

Example of why |Q L| is not a field: the multiplication operation
does not stay within the type:

\begin{code}
  mul : Q L -> Q L -> Q (L *d L) -- |Q Area|
  mul x y = x *q y
\end{code}

\subsection{Measuring quantities}
\label{sec:org89b80dd}

There is one operation missing from |Q d|: measuring the quantity in a
particular system of units. A system of units consists of one scale
factor for each base dimension.
\subsubsection{System of units}
\label{sec:orge6bd6ee}
\begin{code}
  SysUnit : Type
  SysUnit = Vec S 3   -- for the |LTM| class with |n = 3|
\end{code}
We will later implement these two examples:
\begin{spec}
  si   : SysUnit  -- m, kg, second
  cgs  : SysUnit  -- cm, gram, second
\end{spec}

Now we can interpret a "syntactic dimension" (a vector of exponents of
the base dimensions) as its "dimension function" - the corresponding
monomial which computes the change in measured value for a change in
units.

\subsubsection{Dimension function and measure implemented}
\label{sec:orgd24d664}
\begin{code}
  dimfun : Dim -> (SysUnit -> S)
  dimfun d su = foldr _*s_ s1 (zipWith pow su d)
\end{code}

Finally we can define the |measure| of a quantity in a system of
units:

\begin{code}
  measure : Q d -> SysUnit -> S
  measure q su = dimfun (dim q) su *s (LTMQ.val q)
\end{code}

\subsection{Example quantities}
\label{sec:org16b660b}
To make this a bit more concrete we introduce a few quantities:
%if False
\subsubsection{Skip module / opens}
\label{sec:orgbffed95}
\begin{code}
module Examples where
  open LTM Real rfReal
  open LTMDim using (L; T; M)
  open DimStuff dimstuff
  open Qstuff qstuff using (_^q_)
  open LTMQ
\end{code}
%endif
\subsubsection{Length and Acceleration}
\label{sec:orgbce8d66}
\begin{code}
  hei : Q L
  hei = Val 1.78

  Acceleration : Dim
  Acceleration = L /d (T ^d ℤ2)
  g : Q Acceleration
  g = Val 9.82

  si : SysUnit   -- m, kg, second
  si  =   1.0 ∷ 1.0 ∷    1.0 ∷ []
  cgs : SysUnit  -- cm, gram, second
  cgs = 100.0 ∷ 1.0 ∷ 1000.0 ∷ []
\end{code}
\subsubsection{Mass and Force}
\label{sec:org64c2be3}
\begin{code}
  mass : Q M
  mass = Val 76.0

  Force : Dim
  Force = M *d Acceleration
  f : Q Force
  f = mass *q g
\end{code}
\subsubsection{Unit conversion (measure)}
\label{sec:org33dd6cb}
\begin{code}
  test1 : Real
  test1 = measure hei cgs
  check1 : test1 == 178.0
  check1 = refl
  test2 : Real
  test2 = measure f cgs
  check2 : test2 == 7.4632e7
  check2 = refl
\end{code}
\subsection{Dimension analysis (Pi theorem example)}
\label{sec:orgbf60df1}
OK, now we have dimensions, quantities, the dimension function and
some examples. Time to get to the core of dimension analysis: the Pi
theorem.

\subsubsection{Pendulum example intro}
\label{sec:org8d866e3}
\begin{itemize}
\item Assume we are experimenting with an ideal pendulum:
\item a point mass |m| hanging from
\item a piece of string of length |x|,
\item the time |t| for one period
\item when starting from an angle |theta|.
\item We want to find how the gravitational constant |g| can be computed
from the other four parameters.
\end{itemize}

\begin{spec}
  gravity : Q M -> Q L -> Q T -> Q Dimless -> Q Acceleration
  gravity m x t theta = ?
\end{spec}

\subsubsection{Pi theorem for this example}
\label{sec:org4a22469}
\begin{itemize}
\item The core theorem of dimension analysis says that such a function can
be simplified significantly.
\item First, pick three quantities of independent dimension:
here |m|, |x|, and |t|.
\item Second, make the remaining quantities dimensionless by combining
them with monomials in the first three.
\item Here |theta| is already dimensionless, but |g = gravity m x theta t|
is replaced by |a = g *q (t ^q ℤ2) /q x : Q Dimless|.
\item Finally, the function |gravity| is factored into a function |acc|
from just |theta| to |a|, and a monomial factor used to translate
back to a quantity of the correct dimension.
\end{itemize}
\begin{code}
  acc : Q Dimless -> Q Dimless
  gravity : Q M -> Q L -> Q T -> Q Dimless -> Q Acceleration
  gravity m x t theta = monomial  *q  acc theta
    where  monomial : Q Acceleration 
           monomial = x /q (t ^q ℤ2) -- The type dictates the value.
\end{code}
\subsubsection{Pi theorem implications}
\label{sec:orgb2d1435}
\begin{itemize}
\item If we want to figure out the \textbf{4-argument} function |gravity| from
experiments, we actually only need to figure out the \textbf{1-argument}
function |acc|.
\item Experiments (or numerical simulations with the proper differential
equations) further show that the remaining function is constant for
small angles |theta|.
\end{itemize}

\begin{code}
  twoPiSq : Q Dimless
  acc theta = twoPiSq

  twoPiSq = scaleq 39.5 q1
\end{code}

\subsection{Summary \& future work}
\label{sec:org51e8638}
\begin{itemize}
\item Dimension analysis can be usefully expressed done with dep. types
\item Physical quantities form a field graded by an abelian group of
dimensions.
\item Other example: tensors are graded by their rank
\end{itemize}
\subsubsection{{\bfseries\sffamily TODO} Library code and paper is work in progress (mostly in Idris).}
\label{sec:org464cfe2}
\subsubsection{{\bfseries\sffamily TODO} Multiple grading: |T r (Q d s)| or |Q d (T r s)|?}
\label{sec:org77791a8}
\begin{itemize}
\item tensors with rank containing quantities with dimensions?
\item or quantities with dimensions containing tensors with rank?
\item or \ldots{}
\end{itemize}
\end{document}
